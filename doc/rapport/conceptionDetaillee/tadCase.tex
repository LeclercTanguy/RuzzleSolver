
%fonction
%{1}{2}{3}{4}{5}
%(1)Lenomdelafonction
%(2)Lesparam`etres
%(3)Letypederetour
%(4)Lesvariableslocales
%(5)Les instructions

\begin{algorithme}
  \begin{structure}{Case}
    \champStructure{lettre}{a..z}
    \champStructure{nbPoints}{NaturelNonNul}
    \champStructure{bonus}{Bonus}
  \end{structure}

  \fonction
  {creerCase}
  {}
  {Case}
  {case : Case , car : a..z , bonus : CDC}
  {
    \affecter{car}{'a'}
    \affecter{bonus}{'  '}
    fixerLettre(case,car)
    fixerBonus(case,bonus)
    fixerNbPoints(c,1)
    \retouner{case}
  }

  \procedure
  {fixerLettre}
  {\paramEntreeSortie{case : Case} \paramEntree{car : \caractere}}
  {}
  {
    \affecter{c.lettre}{car}
  }

  \procedure
  {fixerNbPoints}
  {\paramEntreeSortie{case : Case} \paramEntree{point : \naturel}}
  {}
  {
    \affecter{c.nbPoints}{point}
  }

  \procedure
  {fixerBonus}
  {\paramEntreeSortie{case : Case} \paramEntree{bonus : Bonus}}
  {}
  {
    \affecter{c.bonus}{bonus}
  }

  \fonction
  {obtenirLettre}
  {case: Case}
  {a..z}
  {}
  {
    \retourner{case.lettre}
  }

  \fonction
  {obtenirNbPoints}
  {case: Case}
  {NaturelNonNul}
  {}
  {
    \retourner{case.nbPoints}
  }

  \fonction
  {obtenirBonus}
  {case: Case}
  {Bonus}
  {}
  {
    \retourner{case.bonus}
  }
