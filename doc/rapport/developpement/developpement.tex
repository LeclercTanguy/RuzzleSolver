\newcommand{\codeC}[1]{
  \inputminted[linenos,breaklines,breakanywhere,xleftmargin=-30pt,xrightmargin=-70pt]
  {C}{#1}\newpage}

\section{Implantation des TAD définis pour le Ruzzle}
  \subsection{Dictionnaire.h}
    \codeC{../../include/Dictionnaire.h}
  \subsection{Mot.h}
    \codeC{../../include/Mot.h}
  \subsection{Case.h}
    \codeC{../../include/Case.h}
  \subsection{CasesContigues.h}
    \codeC{../../include/CasesContigues.h}
  \subsection{Grille.h}
    \codeC{../../include/Grille.h}
  \subsection{Ruzzle.h}
    \codeC{../../include/Ruzzle.h}

\section{Implantation des TAD Collections}
  \subsection{ArbreBinaire.h}
    \codeC{../../include/ArbreBinaire.h}
  \subsection{ABR.h}
    \codeC{../../include/ABR.h}
  \subsection{ListeChainee.h}
    \codeC{../../include/ListeChainee.h}
  \subsection{Ensemble.h}
    \codeC{../../include/Ensemble.h}
  \subsection{tools.h}
    \codeC{../../include/tools.h}

\section{Implémentation des SDD définies pour le Ruzzle}
  \subsection{Dictionnaire.c}
    \codeC{../../src/Dictionnaire.c}
  \subsection{Mot.c}
    \codeC{../../src/Mot.c}
  \subsection{Case.c}
    \codeC{../../src/Case.c}
  \subsection{CasesContigues.c}
    \codeC{../../src/CasesContigues.c}
  \subsection{Grille.c}
    \codeC{../../src/Grille.c}
  \subsection{Ruzzle.c}
    \codeC{../../src/Ruzzle.c}

  \section{Implémentation des SDD Collections}
    \subsection{ArbreBinaire.c}
      \codeC{../../src/ArbreBinaire.c}
    \subsection{ABR.c}
      \codeC{../../src/ABR.c}
    \subsection{ListeChainee.c}
      \codeC{../../src/ListeChainee.c}
    \subsection{Ensemble.h}
      \codeC{../../src/Ensemble.c}
    \subsection{tools.c}
      \codeC{../../src/tools.c}

  \section{ruzzleSolver.c}
    \codeC{../../src/ruzzleSolver.c}
  \section{transcoder.c}
    \codeC{../../src/transcoder.c}

\section{Deuxième version de sauvegarder et charger}
  Le chargement du Dictionnaire (si on exclu sa suppression), représente 97,8\%
  du temps d'exécution de ruzzleSolver, j'ai donc cherché à améliorer sa vitesse
  de chargement pour réduire le temps nécessaire à l'exécution de ruzzleSolver.
  Étant donné que 35\% de ce temps est dédié à l'allocation mémoire (d'après Valgrind),
  j'ai eu l'idée, au lieu d'allouer un espace mémoire pour chaque nœud et chaque
  élément du Dictionnaire, d'allouer un espace de la taille du Dictionnaire entier,
  qui permettrait d'avoir tous les mots dans la même zone mémoire et de ne faire qu'une
  seule allocation.

  J'ai donc modifié la sauvegarde en associant un numéro à chaque nœud afin de connaître
  le positionnement en mémoire du fils gauche et du fils droit.
  Le chargement consistait donc à remplir linéairement cet espace en suivant les données
  générés par l'opération de sauvegarde, le pointeur vers le fils droit étant alors
  l'indice mémoire du premier élément augmenté du numéro du nœud * la taille d'un nœud.
  Chaque nœud contenait donc 3 pointeurs (Dictionnaire basé sur un ArbreBinaire générique)
  ainsi qu'un caractère.

  Le code correspondant à été développé et passe également les tests unitaires.
  Seule la suppression de l'arbre a besoin d'être modifiée car free se réfère à l'espace
  alloué par malloc, on ne peut donc supprimer que l'entièreté de l'arbre.
  La suppression nœud par nœud, ou la suppression d'un mot, ne sont donc plus possible.

  Cette version n'a pas été intégrée au projet final car le chargement du dictionnaire était moins performant. On passe en effet de 0.25s à 0.75s pour l'exécution de ruzzleSolver, ce qui
  est à l'opposé de l'objectif recherché.

  Yves LE GUENNEC
    \codeC{developpement/DictionnaireV2.c}
